
%% bare_jrnl.tex
%% V1.3
%% 2007/01/11
%% by Michael Shell
%% see http://www.michaelshell.org/
%% for current contact information.
%%
%% This is a skeleton file demonstrating the use of IEEEtran.cls
%% (requires IEEEtran.cls version 1.7 or later) with an IEEE journal paper.
%%
%% Support sites:
%% http://www.michaelshell.org/tex/ieeetran/
%% http://www.ctan.org/tex-archive/macros/latex/contrib/IEEEtran/
%% and
%% http://www.ieee.org/



% *** Authors should verify (and, if needed, correct) their LaTeX system  ***
% *** with the testflow diagnostic prior to trusting their LaTeX platform ***
% *** with production work. IEEE's font choices can trigger bugs that do  ***
% *** not appear when using other class files.                            ***
% The testflow support page is at:
% http://www.michaelshell.org/tex/testflow/


%%*************************************************************************
%% Legal Notice:
%% This code is offered as-is without any warranty either expressed or
%% implied; without even the implied warranty of MERCHANTABILITY or
%% FITNESS FOR A PARTICULAR PURPOSE! 
%% User assumes all risk.
%% In no event shall IEEE or any contributor to this code be liable for
%% any damages or losses, including, but not limited to, incidental,
%% consequential, or any other damages, resulting from the use or misuse
%% of any information contained here.
%%
%% All comments are the opinions of their respective authors and are not
%% necessarily endorsed by the IEEE.
%%
%% This work is distributed under the LaTeX Project Public License (LPPL)
%% ( http://www.latex-project.org/ ) version 1.3, and may be freely used,
%% distributed and modified. A copy of the LPPL, version 1.3, is included
%% in the base LaTeX documentation of all distributions of LaTeX released
%% 2003/12/01 or later.
%% Retain all contribution notices and credits.
%% ** Modified files should be clearly indicated as such, including  **
%% ** renaming them and changing author support contact information. **
%%
%% File list of work: IEEEtran.cls, IEEEtran_HOWTO.pdf, bare_adv.tex,
%%                    bare_conf.tex, bare_jrnl.tex, bare_jrnl_compsoc.tex
%%*************************************************************************

% Note that the a4paper option is mainly intended so that authors in
% countries using A4 can easily print to A4 and see how their papers will
% look in print - the typesetting of the document will not typically be
% affected with changes in paper size (but the bottom and side margins will).
% Use the testflow package mentioned above to verify correct handling of
% both paper sizes by the user's LaTeX system.
%
% Also note that the "draftcls" or "draftclsnofoot", not "draft", option
% should be used if it is desired that the figures are to be displayed in
% draft mode.
%
\documentclass[journal]{IEEEtran}
\usepackage[utf8]{inputenc}
\usepackage{blindtext}
\usepackage{graphicx}
\usepackage[numbered,framed]{matlab-prettifier}
\definecolor{mygreen}{RGB}{28,172,0} % color values Red, Green, Blue
\definecolor{mylilas}{RGB}{170,55,241}
\usepackage{listings}
\lstset{
	style=Matlab-editor,
	basicstyle         = \fontsize{8}{11}\ttfamily,
	numberstyle       =\fontsize{8}{11}\ttfamily,
	%backgroundcolor=\color{gray},
	%mlshowsectionrules = true,
	rangeprefix        = \%\ 
}


% Some very useful LaTeX packages include:
% (uncomment the ones you want to load)


% *** MISC UTILITY PACKAGES ***
%
%\usepackage{ifpdf}
% Heiko Oberdiek's ifpdf.sty is very useful if you need conditional
% compilation based on whether the output is pdf or dvi.
% usage:
% \ifpdf
%   % pdf code
% \else
%   % dvi code
% \fi
% The latest version of ifpdf.sty can be obtained from:
% http://www.ctan.org/tex-archive/macros/latex/contrib/oberdiek/
% Also, note that IEEEtran.cls V1.7 and later provides a builtin
% \ifCLASSINFOpdf conditional that works the same way.
% When switching from latex to pdflatex and vice-versa, the compiler may
% have to be run twice to clear warning/error messages.






% *** CITATION PACKAGES ***
%
%\usepackage{cite}
% cite.sty was written by Donald Arseneau
% V1.6 and later of IEEEtran pre-defines the format of the cite.sty package
% \cite{} output to follow that of IEEE. Loading the cite package will
% result in citation numbers being automatically sorted and properly
% "compressed/ranged". e.g., [1], [9], [2], [7], [5], [6] without using
% cite.sty will become [1], [2], [5]--[7], [9] using cite.sty. cite.sty's
% \cite will automatically add leading space, if needed. Use cite.sty's
% noadjust option (cite.sty V3.8 and later) if you want to turn this off.
% cite.sty is already installed on most LaTeX systems. Be sure and use
% version 4.0 (2003-05-27) and later if using hyperref.sty. cite.sty does
% not currently provide for hyperlinked citations.
% The latest version can be obtained at:
% http://www.ctan.org/tex-archive/macros/latex/contrib/cite/
% The documentation is contained in the cite.sty file itself.






% *** GRAPHICS RELATED PACKAGES ***
%
\ifCLASSINFOpdf
  % \usepackage[pdftex]{graphicx}
  % declare the path(s) where your graphic files are
  % \graphicspath{{../pdf/}{../jpeg/}}
  % and their extensions so you won't have to specify these with
  % every instance of \includegraphics
  % \DeclareGraphicsExtensions{.pdf,.jpeg,.png}
\else
  % or other class option (dvipsone, dvipdf, if not using dvips). graphicx
  % will default to the driver specified in the system graphics.cfg if no
  % driver is specified.
  % \usepackage[dvips]{graphicx}
  % declare the path(s) where your graphic files are
  % \graphicspath{{../eps/}}
  % and their extensions so you won't have to specify these with
  % every instance of \includegraphics
  % \DeclareGraphicsExtensions{.eps}
\fi
% graphicx was written by David Carlisle and Sebastian Rahtz. It is
% required if you want graphics, photos, etc. graphicx.sty is already
% installed on most LaTeX systems. The latest version and documentation can
% be obtained at: 
% http://www.ctan.org/tex-archive/macros/latex/required/graphics/
% Another good source of documentation is "Using Imported Graphics in
% LaTeX2e" by Keith Reckdahl which can be found as epslatex.ps or
% epslatex.pdf at: http://www.ctan.org/tex-archive/info/
%
% latex, and pdflatex in dvi mode, support graphics in encapsulated
% postscript (.eps) format. pdflatex in pdf mode supports graphics
% in .pdf, .jpeg, .png and .mps (metapost) formats. Users should ensure
% that all non-photo figures use a vector format (.eps, .pdf, .mps) and
% not a bitmapped formats (.jpeg, .png). IEEE frowns on bitmapped formats
% which can result in "jaggedy"/blurry rendering of lines and letters as
% well as large increases in file sizes.
%
% You can find documentation about the pdfTeX application at:
% http://www.tug.org/applications/pdftex





% *** MATH PACKAGES ***
%
\usepackage[cmex10]{amsmath}
% A popular package from the American Mathematical Society that provides
% many useful and powerful commands for dealing with mathematics. If using
% it, be sure to load this package with the cmex10 option to ensure that
% only type 1 fonts will utilized at all point sizes. Without this option,
% it is possible that some math symbols, particularly those within
% footnotes, will be rendered in bitmap form which will result in a
% document that can not be IEEE Xplore compliant!
%
% Also, note that the amsmath package sets \interdisplaylinepenalty to 10000
% thus preventing page breaks from occurring within multiline equations. Use:
\interdisplaylinepenalty=2500
% after loading amsmath to restore such page breaks as IEEEtran.cls normally
% does. amsmath.sty is already installed on most LaTeX systems. The latest
% version and documentation can be obtained at:
% http://www.ctan.org/tex-archive/macros/latex/required/amslatex/math/





% *** SPECIALIZED LIST PACKAGES ***
%
%\usepackage{algorithmic}
% algorithmic.sty was written by Peter Williams and Rogerio Brito.
% This package provides an algorithmic environment fo describing algorithms.
% You can use the algorithmic environment in-text or within a figure
% environment to provide for a floating algorithm. Do NOT use the algorithm
% floating environment provided by algorithm.sty (by the same authors) or
% algorithm2e.sty (by Christophe Fiorio) as IEEE does not use dedicated
% algorithm float types and packages that provide these will not provide
% correct IEEE style captions. The latest version and documentation of
% algorithmic.sty can be obtained at:
% http://www.ctan.org/tex-archive/macros/latex/contrib/algorithms/
% There is also a support site at:
% http://algorithms.berlios.de/index.html
% Also of interest may be the (relatively newer and more customizable)
% algorithmicx.sty package by Szasz Janos:
% http://www.ctan.org/tex-archive/macros/latex/contrib/algorithmicx/




% *** ALIGNMENT PACKAGES ***
%
%\usepackage{array}
% Frank Mittelbach's and David Carlisle's array.sty patches and improves
% the standard LaTeX2e array and tabular environments to provide better
% appearance and additional user controls. As the default LaTeX2e table
% generation code is lacking to the point of almost being broken with
% respect to the quality of the end results, all users are strongly
% advised to use an enhanced (at the very least that provided by array.sty)
% set of table tools. array.sty is already installed on most systems. The
% latest version and documentation can be obtained at:
% http://www.ctan.org/tex-archive/macros/latex/required/tools/


%\usepackage{mdwmath}
%\usepackage{mdwtab}
% Also highly recommended is Mark Wooding's extremely powerful MDW tools,
% especially mdwmath.sty and mdwtab.sty which are used to format equations
% and tables, respectively. The MDWtools set is already installed on most
% LaTeX systems. The lastest version and documentation is available at:
% http://www.ctan.org/tex-archive/macros/latex/contrib/mdwtools/


% IEEEtran contains the IEEEeqnarray family of commands that can be used to
% generate multiline equations as well as matrices, tables, etc., of high
% quality.


%\usepackage{eqparbox}
% Also of notable interest is Scott Pakin's eqparbox package for creating
% (automatically sized) equal width boxes - aka "natural width parboxes".
% Available at:
% http://www.ctan.org/tex-archive/macros/latex/contrib/eqparbox/





% *** SUBFIGURE PACKAGES ***
\usepackage[tight,footnotesize]{subfigure}
% subfigure.sty was written by Steven Douglas Cochran. This package makes it
% easy to put subfigures in your figures. e.g., "Figure 1a and 1b". For IEEE
% work, it is a good idea to load it with the tight package option to reduce
% the amount of white space around the subfigures. subfigure.sty is already
% installed on most LaTeX systems. The latest version and documentation can
% be obtained at:
% http://www.ctan.org/tex-archive/obsolete/macros/latex/contrib/subfigure/
% subfigure.sty has been superceeded by subfig.sty.



%\usepackage[caption=false]{caption}
%\usepackage[font=footnotesize]{subfig}
% subfig.sty, also written by Steven Douglas Cochran, is the modern
% replacement for subfigure.sty. However, subfig.sty requires and
% automatically loads Axel Sommerfeldt's caption.sty which will override
% IEEEtran.cls handling of captions and this will result in nonIEEE style
% figure/table captions. To prevent this problem, be sure and preload
% caption.sty with its "caption=false" package option. This is will preserve
% IEEEtran.cls handing of captions. Version 1.3 (2005/06/28) and later 
% (recommended due to many improvements over 1.2) of subfig.sty supports
% the caption=false option directly:
%\usepackage[caption=false,font=footnotesize]{subfig}
%
% The latest version and documentation can be obtained at:
% http://www.ctan.org/tex-archive/macros/latex/contrib/subfig/
% The latest version and documentation of caption.sty can be obtained at:
% http://www.ctan.org/tex-archive/macros/latex/contrib/caption/




% *** FLOAT PACKAGES ***
%
%\usepackage{fixltx2e}
% fixltx2e, the successor to the earlier fix2col.sty, was written by
% Frank Mittelbach and David Carlisle. This package corrects a few problems
% in the LaTeX2e kernel, the most notable of which is that in current
% LaTeX2e releases, the ordering of single and double column floats is not
% guaranteed to be preserved. Thus, an unpatched LaTeX2e can allow a
% single column figure to be placed prior to an earlier double column
% figure. The latest version and documentation can be found at:
% http://www.ctan.org/tex-archive/macros/latex/base/



%\usepackage{stfloats}
% stfloats.sty was written by Sigitas Tolusis. This package gives LaTeX2e
% the ability to do double column floats at the bottom of the page as well
% as the top. (e.g., "\begin{figure*}[!b]" is not normally possible in
% LaTeX2e). It also provides a command:
%\fnbelowfloat
% to enable the placement of footnotes below bottom floats (the standard
% LaTeX2e kernel puts them above bottom floats). This is an invasive package
% which rewrites many portions of the LaTeX2e float routines. It may not work
% with other packages that modify the LaTeX2e float routines. The latest
% version and documentation can be obtained at:
% http://www.ctan.org/tex-archive/macros/latex/contrib/sttools/
% Documentation is contained in the stfloats.sty comments as well as in the
% presfull.pdf file. Do not use the stfloats baselinefloat ability as IEEE
% does not allow \baselineskip to stretch. Authors submitting work to the
% IEEE should note that IEEE rarely uses double column equations and
% that authors should try to avoid such use. Do not be tempted to use the
% cuted.sty or midfloat.sty packages (also by Sigitas Tolusis) as IEEE does
% not format its papers in such ways.


%\ifCLASSOPTIONcaptionsoff
%  \usepackage[nomarkers]{endfloat}
% \let\MYoriglatexcaption\caption
% \renewcommand{\caption}[2][\relax]{\MYoriglatexcaption[#2]{#2}}
%\fi
% endfloat.sty was written by James Darrell McCauley and Jeff Goldberg.
% This package may be useful when used in conjunction with IEEEtran.cls'
% captionsoff option. Some IEEE journals/societies require that submissions
% have lists of figures/tables at the end of the paper and that
% figures/tables without any captions are placed on a page by themselves at
% the end of the document. If needed, the draftcls IEEEtran class option or
% \CLASSINPUTbaselinestretch interface can be used to increase the line
% spacing as well. Be sure and use the nomarkers option of endfloat to
% prevent endfloat from "marking" where the figures would have been placed
% in the text. The two hack lines of code above are a slight modification of
% that suggested by in the endfloat docs (section 8.3.1) to ensure that
% the full captions always appear in the list of figures/tables - even if
% the user used the short optional argument of \caption[]{}.
% IEEE papers do not typically make use of \caption[]'s optional argument,
% so this should not be an issue. A similar trick can be used to disable
% captions of packages such as subfig.sty that lack options to turn off
% the subcaptions:
% For subfig.sty:
% \let\MYorigsubfloat\subfloat
% \renewcommand{\subfloat}[2][\relax]{\MYorigsubfloat[]{#2}}
% For subfigure.sty:
% \let\MYorigsubfigure\subfigure
% \renewcommand{\subfigure}[2][\relax]{\MYorigsubfigure[]{#2}}
% However, the above trick will not work if both optional arguments of
% the \subfloat/subfig command are used. Furthermore, there needs to be a
% description of each subfigure *somewhere* and endfloat does not add
% subfigure captions to its list of figures. Thus, the best approach is to
% avoid the use of subfigure captions (many IEEE journals avoid them anyway)
% and instead reference/explain all the subfigures within the main caption.
% The latest version of endfloat.sty and its documentation can obtained at:
% http://www.ctan.org/tex-archive/macros/latex/contrib/endfloat/
%
% The IEEEtran \ifCLASSOPTIONcaptionsoff conditional can also be used
% later in the document, say, to conditionally put the References on a 
% page by themselves.





% *** PDF, URL AND HYPERLINK PACKAGES ***
%
%\usepackage{url}
% url.sty was written by Donald Arseneau. It provides better support for
% handling and breaking URLs. url.sty is already installed on most LaTeX
% systems. The latest version can be obtained at:
% http://www.ctan.org/tex-archive/macros/latex/contrib/misc/
% Read the url.sty source comments for usage information. Basically,
% \url{my_url_here}.





% *** Do not adjust lengths that control margins, column widths, etc. ***
% *** Do not use packages that alter fonts (such as pslatex).         ***
% There should be no need to do such things with IEEEtran.cls V1.6 and later.
% (Unless specifically asked to do so by the journal or conference you plan
% to submit to, of course. )


% correct bad hyphenation here
\hyphenation{op-tical net-works semi-conduc-tor}


\begin{document}
%
% paper title
% can use linebreaks \\ within to get better formatting as desired
\title{Recherche Operationnelle\\Modélisation Paramétrique,
	Filtrage Optimal\\ et Adaptatif\\Introduction au Data Mining
	Réseaux de Neurones
}
%
%
% author names and IEEE memberships
% note positions of commas and nonbreaking spaces ( ~ ) LaTeX will not break
% a structure at a ~ so this keeps an author's name from being broken across
% two lines.
% use \thanks{} to gain access to the first footnote area
% a separate \thanks must be used for each paragraph as LaTeX2e's \thanks
% was not built to handle multiple paragraphs
%

\author{Mauricio~Caceres,~\IEEEmembership{Master,~SISEA,~ENSSAT,~Lannion}
      %  and~Jane~Doe,~\IEEEmembership{Life~Fellow,~IEEE}% <-this % stops a space
\thanks{M. Pascal Scalart,~Pôle Electronique, Enssat, Lannion,
France  e-mail: (pascal.scalart@univ-rennes1.fr).}}% <-this % stops a space
%\thanks{J. Doe and J. Doe are with Anonymous University.}% <-this % stops a space
%\thanks{Manuscript received April 19, 2005; revised January 11, 2007.}}

% note the % following the last \IEEEmembership and also \thanks - 
% these prevent an unwanted space from occurring between the last author name
% and the end of the author line. i.e., if you had this:
% 
% \author{....lastname \thanks{...} \thanks{...} }
%                     ^------------^------------^----Do not want these spaces!
%
% a space would be appended to the last name and could cause every name on that
% line to be shifted left slightly. This is one of those "LaTeX things". For
% instance, "\textbf{A} \textbf{B}" will typeset as "A B" not "AB". To get
% "AB" then you have to do: "\textbf{A}\textbf{B}"
% \thanks is no different in this regard, so shield the last } of each \thanks
% that ends a line with a % and do not let a space in before the next \thanks.
% Spaces after \IEEEmembership other than the last one are OK (and needed) as
% you are supposed to have spaces between the names. For what it is worth,
% this is a minor point as most people would not even notice if the said evil
% space somehow managed to creep in.



% The paper headers
\markboth{Journal of \LaTeX\ Class Files,~Vol.~6, No.~1, January~2007}%
{Shell \MakeLowercase{\textit{et al.}}: Bare Demo of IEEEtran.cls for Journals}
% The only time the second header will appear is for the odd numbered pages
% after the title page when using the twoside option.
% 
% *** Note that you probably will NOT want to include the author's ***
% *** name in the headers of peer review papers.                   ***
% You can use \ifCLASSOPTIONpeerreview for conditional compilation here if
% you desire.




% If you want to put a publisher's ID mark on the page you can do it like
% this:
%\IEEEpubid{0000--0000/00\$00.00~\copyright~2007 IEEE}
% Remember, if you use this you must call \IEEEpubidadjcol in the second
% column for its text to clear the IEEEpubid mark.



% use for special paper notices
%\IEEEspecialpapernotice{(Invited Paper)}




% make the title area
\maketitle


\begin{abstract}
%\boldmath
Ce projet nous amenne à découvrir et faire une première application des concepts de 
machine learning. Les concepts de machine learning, deep learning, et réseau de neurones
sont très couramment nommes. Sont très utilises pour la reconnaisances d'"images et la ré"slution
de problèmes difficiles à modéliser et aussi avec une mathematique difficile à résoudre.
Les techniques de machine learning ont une nature très empirique mais aussi une partie teorique
et des fondaments mathematiques.\\
Dans ce rapport les résultats de l'implementation d'une réseau des neurones multi-couches seront 
détailles avec differentes notions qui permetent de fondamenter les prise de décision au niveuau de 
l'implementation.

\end{abstract}
% IEEEtran.cls defaults to using nonbold math in the Abstract.
% This preserves the distinction between vectors and scalars. However,
% if the journal you are submitting to favors bold math in the abstract,
% then you can use LaTeX's standard command \boldmath at the very start
% of the abstract to achieve this. Many IEEE journals frown on math
% in the abstract anyway.

% Note that keywords are not normally used for peerreview papers.
\begin{IEEEkeywords}
Réseaux de Neurones, perceptron multi-couches, convergence, backpropagation, descente maximale
\end{IEEEkeywords}


% For peer review papers, you can put extra information on the cover
% page as needed:
% \ifCLASSOPTIONpeerreview
% \begin{center} \bfseries EDICS Category: 3-BBND \end{center}
% \fi
%
% For peerreview papers, this IEEEtran command inserts a page break and
% creates the second title. It will be ignored for other modes.
\IEEEpeerreviewmaketitle



\section{Introduction}
\blindtext
\begin{figure}[h]
	\centering
	\includegraphics[width=2.5in]{logo}
	%TODO change the image and the caption and label
	\caption{Simulation Results}
	\label{fig_sim}
\end{figure}
\subsection{Objectifs}
\blindtext

% needed in second column of first page if using \IEEEpubid
%\IEEEpubidadjcol

% An example of a floating figure using the graphicx package.
% Note that \label must occur AFTER (or within) \caption.
% For figures, \caption should occur after the \includegraphics.
% Note that IEEEtran v1.7 and later has special internal code that
% is designed to preserve the operation of \label within \caption
% even when the captionsoff option is in effect. However, because
% of issues like this, it may be the safest practice to put all your
% \label just after \caption rather than within \caption{}.
%
% Reminder: the "draftcls" or "draftclsnofoot", not "draft", class
% option should be used if it is desired that the figures are to be
% displayed while in draft mode.
%
%\begin{figure}[!t]
%\centering
%\includegraphics[width=2.5in]{myfigure}
% where an .eps filename suffix will be assumed under latex, 
% and a .pdf suffix will be assumed for pdflatex; or what has been declared
% via \DeclareGraphicsExtensions.
%\caption{Simulation Results}
%\label{fig_sim}
%\end{figure}

% Note that IEEE typically puts floats only at the top, even when this
% results in a large percentage of a column being occupied by floats.


% An example of a double column floating figure using two subfigures.
% (The subfig.sty package must be loaded for this to work.)
% The subfigure \label commands are set within each subfloat command, the
% \label for the overall figure must come after \caption.
% \hfil must be used as a separator to get equal spacing.
% The subfigure.sty package works much the same way, except \subfigure is
% used instead of \subfloat.
%
%\begin{figure*}[!t]
%\centerline{\subfloat[Case I]\includegraphics[width=2.5in]{subfigcase1}%
%\label{fig_first_case}}
%\hfil
%\subfloat[Case II]{\includegraphics[width=2.5in]{subfigcase2}%
%\label{fig_second_case}}}
%\caption{Simulation results}
%\label{fig_sim}
%\end{figure*}
%
% Note that often IEEE papers with subfigures do not employ subfigure
% captions (using the optional argument to \subfloat), but instead will
% reference/describe all of them (a), (b), etc., within the main caption.


% An example of a floating table. Note that, for IEEE style tables, the 
% \caption command should come BEFORE the table. Table text will default to
% \footnotesize as IEEE normally uses this smaller font for tables.
% The \label must come after \caption as always.
%
%\begin{table}[!t]
%% increase table row spacing, adjust to taste
%\renewcommand{\arraystretch}{1.3}
% if using array.sty, it might be a good idea to tweak the value of
% \extrarowheight as needed to properly center the text within the cells
%\caption{An Example of a Table}
%\label{table_example}
%\centering
%% Some packages, such as MDW tools, offer better commands for making tables
%% than the plain LaTeX2e tabular which is used here.
%\begin{tabular}{|c||c|}
%\hline
%One & Two\\
%\hline
%Three & Four\\
%\hline
%\end{tabular}
%\end{table}


% Note that IEEE does not put floats in the very first column - or typically
% anywhere on the first page for that matter. Also, in-text middle ("here")
% positioning is not used. Most IEEE journals use top floats exclusively.
% Note that, LaTeX2e, unlike IEEE journals, places footnotes above bottom
% floats. This can be corrected via the \fnbelowfloat command of the
% stfloats package.


%%%%%%%%%%%%%%%%%%%%%%%%%%%%%%%%%%%%%%%%%%%%%%%%%%%%%%%%
%
%			YOUR CONTENT HERE 
%
%%%%%%%%%%%%%%%%%%%%%%%%%%%%%%%%%%%%%%%%%%%%%%%%%%%%%%%%

\section{Modélisation d’un neurone}
Le modèle utilisé dans ce cas, inspiré dans le fonctionnement de une neurone biologique est le perceptron.
%todo add ssome history
\begin{figure}[h]
	\centering
	\includegraphics[width=2.5in]{perceptron_modelo}
	%TODO change the image tiene amarillo aca nada que ver loco
	\caption{Modèle de perceptron utilise dans le réseau de neurones}
	\label{fig:perceptron_modelo}
\end{figure}
Les éléments suivant sont partie du modèle et necessaire pour comprendre le fonctionnement du perceptron.
\begin{itemize}
	\item nombre de signaux d'entrée $ x_0,...,x_{N-1} $
	\item poids de connexions $w_{j,0},...,w_{j,L-1} $
	\item fonction d'activation $v = k(w,x) $
	\item fonction de transition $ f $
	 \item un état de sortie  $s_j=f(v_j) $
	 \item un entrée fixe: le bias
\end{itemize}
%todo mejorar redaction sobre la function sigmoided
On a utilisé la fonction sigmoide comme fonction de transition. La fonction de activation est la fonction produit escalaire. Les connections entre neurones est faite par le bias du poids. Si la connections a un poid negative est une connection dite \textit{exhibitrice} si est positif est dite \textit{excitatrice}. C'est une manière de modeliser la force de la connection synaptique entre ce perceptron là et une autre à lequel est connecté.


\subsection{Le perceptron : fonctions de transition classiques}
Ce sont les fonctions linaire, sigmoide, tangente hyperbolique ou autres.Cette fonction nous permets avoir la valeur de sortie de la neurone, si elle est positive est dite une neurone actif. Dans le cas contraire es dit inactif.
Pour notre cas la function sigmoide de la Fig, \ref{fig:sigmoide} c'est la que nous permet avoir des valeur entre 0 et 1. Cela permettra de fixer le seuil d'activité de la neurone j.


\begin{figure}[h]
	\centering
	\includegraphics[width=2.5in]{sigmoide}
	%TODO change the image elle est pourri
	\caption{Fonction sigmoide et sa derivé}
	\label{fig:sigmoide}
\end{figure}

\subsection{Le perceptron : rôle de l’unité de biais}
Dans la figure \ref{fig:perceptron_modelo} on peut voir que la premier entrée $ w_{j,0} $ es toujours à -1. La fonction 
de cette poids qu'on appelle bias est de fixer le seuil d'activation de la neurone. Permettant de déplacer l'hyperplan 
de la fonction d'activation qui peut-être vu comme une modification de la façon dont la neurone va se comporté face aux données, elle va classifier de manière différente



\section{Le perceptron  multicouche}
À cause de l'impossibilité du perceptron  pour classifier dans certain cas la connection de une couche
de neurones a une autre supplémentaires nous donne une 



\subsection{Architecture utilise}

\begin{figure}[h]
	\centering
	\includegraphics[width=2.5in]{arch}
	%TODO change the image and the caption and label
	\caption{Architecture de la réseaux utilisé. MLP conventional }
	\label{fig:ArchitectureMLP}
\end{figure}


\subsection{L’erreur du perceptron : une fonction multidimensionelle}\blindtext

\begin{figure}[h]
	\centering
	\includegraphics[width=2in]{error}
	%TODO change the image and the caption and label
	\caption{Courbes indicant l'erreur d'apprentissage (risque empirique) et l'erreur de test(risque réel)}
	\label{fig:courbesError}
\end{figure}

\section{L’apprentissage supervisé}
Dans le programme du projet on a implimenté une processus d'apprentissage supervisé. Ce algorithme a 
comme finalité la modificatioon des poids de la couche d"entreé et de sortie permettant de l'adapter aux 
entrées d'accord à la fonction de coût.
Il s'agit de presenter dans l'entrée des examples $x(n)$ correspondant à la base de données de entrenaiment et 
ensuite comparer la sortie $ y(n) $ obtenue à la sortie theorique (classe correcte dans le cadre de la classification) ou correcte. Le fait de savoir à priori lequelle est la sortie correcte fait que l'algorithme
soit supervisé.\\
La modification des poids initiaux  va nous amener vers une réseaux entrainé correctement lorsque l'erreur 
diminue.
Notre strategie a été de presenter les examples à l'entrée  $x(n)$ selon le type de son, toujours en respectant l'ordre son 1, son 2, son 3. De cette façon on connais toujours quelle classe de son on a en
entrée pour après calculer l'erreur.


\section{Algorithme BackPropagation : calcul du vecteur gradient}\blindtext

Avec ce type de réseau on peut traiter different types d'information comme signaux (images, sons). 
Sont capables de résoudre des problèmes de classification par le bias d'un processus d"apprentissage.
L'algorithme d'apprentissage es s'agite de 


%%%%%%%%%%%%%%%%%%%%%%%%%%%
%
% 	DOCUMENT LE PROGRAMME RS(SIMPLE)
%
%%%%%%%%%%%%%%%%%%%%%%%%%%%
\section{Cas d’un perceptron multicouche : implementations}
Dans cette section on parlera sur les détails d'implementation de notre réseaux multi-couche. Les strategies
suivis et les choix de parametres et de techniques de convergence. Il y a quatre codes avec des implementations et techniques differentes. On montrera leur performance et comportement. A la fin on aura 
des conclusions.


\subsection{Base d'apprentissage et de test}
L'apprentissage va se faire sur les caractéristiques plus fortes de notre données, soit des signaux ou des images ou n'importe quelle autre type de donne. Mais il faut trouver un moyen de représenter ou  les extraire. On appelle attributs à les descriptors de la donnée\\
Aussi pour des limitations pratiques et d'implementation cette répresentation doit être échantillonné.
Pour notre application on s'en sert de l'estimation de la DSP (densité spectrale de puissance) du signal à l’aide de la méthode du périodogramme fenêtré.

\begin{lstlisting}[caption={Code puor initialisation des variables},label=code_initial]
[dataTest,fs,Nbits] = wavread(name);
% calcul des data_test
L = size(dataTest,1);
X = fft(dataTest.*hamming(L),Nfft);
Sxx = 1/Nfft*abs(X).^2;
data_test(:,in) = Sxx(1:size(data_test,1));
\end{lstlisting}

%todo escribir algo aca sobre las se;ales

\begin{figure}[h]
	\centering
	\includegraphics[width=2.5in]{../OctaveNeurons/rs5}
	\caption{Comparation entre les attributs normalisées,non normalisés}
	\label{fig:AttributsNorm}
\end{figure}


\subsection{Parametrage du MLP}

Notre MLP a 4098 entrées dans la couche d'entrée, connectes chaqu'une a 100 neurones dans la couche cachée. 
Dans la sortie intermediaire on a donc 100 sortie connectés à l'entrée du la couche de sortie qui a 3 neurones. La sortie de chacune des trois dernières neurones sont la sortie de notre réseaux.\\
Le taux d'apprentissage nous donne selon l'algorithme de backpropagation le rythme à lequel les poids sont
modifies pour s'ajuster aux entrées (fit data).

\begin{lstlisting}
L_in = 4097+1;
L_cachee = 100+1;
L_out = 3;
mu = 0.5 %taux d'apprentissage
a=-0.5;
b = 0.5;
C = [ a + (b-a).*rand(L_in,L_cachee)];
W = [a + (b-a).*rand(L_cachee,L_out)];
\end{lstlisting}
 Le  parametre $\mu$ est tres importante parce que modifie aussi 
la duree de la algortihme pour convergence. Si on converge tres rapidement avec un taux d'apprentisage grand
on ne va pas  bien apprendre la base de données donc on a le risque d'avoir une basse performance au niveau 
de la distinctions des examples de test parce que la classification est très
generaliste. Si le taux d'apprentissage es plus petit l'entrainement prendera plus
de temps et ça faire que le réseaux apprenne
beacoup mieux les examples. Par contre si on entraine beacoup les neurones il y a
le risque qu'elle aprenne 
la base de données tel comme elle est, en commetant ce que on appele
\textit{overfitting} ou \textit{surapprentissage}. Dans ce cas il faut appliquer
des techniques especifiques pour obtenir des résultats differentes.On parlera plus
tard sur ce sujet là.\\
%todo agregar una referencia a el surapprentisage et aussi a 
\begin{figure}[h]
	\centering
	\includegraphics[width=2.5in]{../OctaveNeurons/rs3}
	\caption{Histogramme des poids de la couche de sortie }
	\label{fig:histoW}
\end{figure}
Les matrices C et W ont le poids du connections entre neurones et entrées et
sorties. Dans la Fig.\ref{fig:histoW} et la Fig.\ref{fig:histoC} on voit les 
histogrammes des valeurs des poids pour les connections d'une neurone.

\begin{figure}[h]
	\centering
	\includegraphics[width=2.5in]{../OctaveNeurons/rs4}
	\caption{Histogramme des poids de la couche de cachée}
	\label{fig:histoC}
\end{figure}

\subsection{Test sur une version simple du MLP}

\begin{figure}[h]
	\centering
	\includegraphics[width=2.5in]{../OctaveNeurons/rs2}
	\caption{Graphique pour comparation des poids de sortie initiaux et finales}
	\label{fig:SignauxNormalises}
\end{figure}

Les courbes les plus importantes sont celles ci que nous montrent comment se comporte
l'erreur. L'erreur commence au départ à diminuer et ensuite commence à ce stabiliser.
Le critère d'arrêt de l'algorithme est arriver à un'erreur plus petit que 1e-4.
\begin{figure}[h]
	\centering
	\includegraphics[width=2.5in]{../OctaveNeurons/rs1}
	\caption{Résultats de implementation simple MLP, $\mu=0.5$,initialisation poids non normalement distribuée}
	\label{fig:CourbeError}
\end{figure}

\section{Amelioration de l'algorithme pour une meuilleur convergence}
Entre plusiers techniques pour faire avoir une convergence dans une période des temps aceptable.
\begin{itemize}
	\item permutation des examples de la base de données
	\item même frequence pour classe
	\item normalisation de l'entrée
\end{itemize}
Tous ces astuces sont déjà implementes dans la version base du code. Dans les programmes suivantes on
a la mise en place de :
\begin{itemize}
	\item rs2 : initialisation des poids avec distribution normal
	\item rs3 : On rajoute un taux d'apprentissage variable
	\item rs4 : On rajoute deux taux d'apprentissage variable
\end{itemize}

\section{Programme rs2 : initialisation des poids avec distribution normal}\blindtext

%%%%%%%%%%%%%%%%%%%%%%%%%%%
% 	DOCUMENT LE PROGRAMME RS2
%%%%%%%%%%%%%%%%%%%%%%%%%%%
La initialisation des poids de les matrices W et C sont fais à l'aide d'une function qui genere
des numeros aleatoires avec une distribution normal centrée et paramètre $ \sigma=\sqrt{m} $
en étant m les nombres de connection avec la couche précédente.
\begin{lstlisting}
L_in = 4097+1;
L_cachee = 100+1;
L_out = 3;

%sigmas de distrib. de proba
sigma1=1/sqrt(L_in);
sigma2 = 1/sqrt(L_cachee);

%creation the matrices de poids
C = normrnd (0, sigma1, L_in, L_cachee);
W =   normrnd(0, sigma2, L_cachee, L_out);	
\end{lstlisting}


\begin{figure}[h]
	\centering
	\includegraphics[width=2.5in]{../OctaveNeurons/rs22}
	%TODO change the image and the caption and label
	\caption{Histogramme des poids de la couche de entrée d'une neurone.Distribution Normal }
\label{fig:histoW}
\end{figure}
\blindtext

\begin{figure}[h]
	\centering
	\includegraphics[width=2.5in]{../OctaveNeurons/rs23}
	%TODO change the image and the caption and label
	\caption{Histogramme des poids de la couche de sortie d'une neurone.Distribution Normal }
\label{fig:histoW}
\end{figure}

\begin{lstlisting}
rs2 avec mu = 0.5

error =    9.9677e-05
error_test =  0.014817
elapsed_time =  37.162
----------------------
rs2 avec mu = 0.9

error =    9.9585e-05
error_test =  0.019871
elapsed_time =  26.649
-----------------------
rs2 avec mu = 0.3

error =    9.9595e-05
error_test =  0.0095784
elapsed_time =  73.670
\end{lstlisting}

Voici la courbe d'erreur d'apprentisage et de test pour une valeur de $\mu=0.5$ 
\begin{figure}[h]
	\centering
	\includegraphics[width=2.5in]{../OctaveNeurons/rs21}
	%TODO change the image and the caption and label
	\caption{Simulation Results}
	\label{fig_sim}
\end{figure}


\section{Programme rs3 : On rajoute un taux d'apprentissage variable}\blindtext

A cette étape là on fait varier le taux d'apprentissage
%%%%%%%%%%%%%%%%%%%%%%%%%%%
%
% 	DOCUMENT LE PROGRAMME RS3
%
%%%%%%%%%%%%%%%%%%%%%%%%%%%

\begin{lstlisting}
%Init taux d'apprentissage
alpha = 0.5;
muo = 0.5;
mu = muo/(1+alpha*1)
\end{lstlisting}
Et dans la boucle principal on fait la mise à jour
\begin{lstlisting}
    for i = 1:60 %parcourir la base de donnees
	    ...
	    ...
	    %calcul des sorties, erreur, update poids
	    ...
	    ...
		%update mu
		mu = muo/(1+alpha*i);
		storemu(iter)=mu; %store for graphs
	end
\end{lstlisting}
\blindtext
\begin{figure}[h]
	\centering
	\includegraphics[width=2.5in]{../OctaveNeurons/rs31}
	%TODO change the image and the caption and label
	\caption{Simulation Results}
	\label{fig_sim}
\end{figure}
\blindtext


\section{Programme Rs4 : On rajoute deux taux d'apprentissage variable}\blindtext
%%%%%%%%%%%%%%%%%%%%%%%%%%%
%
% 	DOCUMENT LE PROGRAMME RS4
%
%%%%%%%%%%%%%%%%%%%%%%%%%%%
\begin{figure}[h]
	\centering
	\includegraphics[width=2.5in]{../OctaveNeurons/rs31}
	%TODO change the image and the caption and label
	\caption{Simulation Results}
	\label{fig_sim}
\end{figure}
\blindtext


\section{Conclusion}

Dans cette projet on a vu par le bias de l'implementation de un réseaux de neurones, les 
effet consequence de la manipulation des differentes parametres du calcul.\\
La partie empirique est très importante parce que 


%%%%%%%%%%%%%%%%%%%%%%%%%%%%%%%%%%%%%%%%%%%%%%%%%%%%%%%%
%
%			ALL THE SCIENTIFIC'S STUFF FINISH HERE
%
%%%%%%%%%%%%%%%%%%%%%%%%%%%%%%%%%%%%%%%%%%%%%%%%%%%%%%%%
% if have a single appendix:
%\appendix[Proof of the Zonklar Equations]
% or
%\appendix  % for no appendix heading
% do not use \section anymore after \appendix, only \section*
% is possibly needed

% use appendices with more than one appendix
% then use \section to start each appendix
% you must declare a \section before using any
% \subsection or using \label (\appendices by itself
% starts a section numbered zero.)
%


\appendices
\section{Proof of the First Zonklar Equation}
Some text for the appendix.

% use section* for acknowledgement
\section*{Acknowledgment}


The authors would like to thank...


% Can use something like this to put references on a page
% by themselves when using endfloat and the captionsoff option.
\ifCLASSOPTIONcaptionsoff
  \newpage
\fi



% trigger a \newpage just before the given reference
% number - used to balance the columns on the last page
% adjust value as needed - may need to be readjusted if
% the document is modified later
%\IEEEtriggeratref{8}
% The "triggered" command can be changed if desired:
%\IEEEtriggercmd{\enlargethispage{-5in}}

% references section

% can use a bibliography generated by BibTeX as a .bbl file
% BibTeX documentation can be easily obtained at:
% http://www.ctan.org/tex-archive/biblio/bibtex/contrib/doc/
% The IEEEtran BibTeX style support page is at:
% http://www.michaelshell.org/tex/ieeetran/bibtex/
%\bibliographystyle{IEEEtran}
% argument is your BibTeX string definitions and bibliography database(s)
%\bibliography{IEEEabrv,../bib/paper}
%
% <OR> manually copy in the resultant .bbl file
% set second argument of \begin to the number of references
% (used to reserve space for the reference number labels box)
\begin{thebibliography}{1}

\bibitem{IEEEhowto:kopka}
H.~Kopka and P.~W. Daly, \emph{A Guide to \LaTeX}, 3rd~ed.\hskip 1em plus
  0.5em minus 0.4em\relax Harlow, England: Addison-Wesley, 1999.

\end{thebibliography}

% biography section
% 
% If you have an EPS/PDF photo (graphicx package needed) extra braces are
% needed around the contents of the optional argument to biography to prevent
% the LaTeX parser from getting confused when it sees the complicated
% \includegraphics command within an optional argument. (You could create
% your own custom macro containing the \includegraphics command to make things
% simpler here.)
%\begin{biography}[{\includegraphics[width=1in,height=1.25in,clip,keepaspectratio]{mshell}}]{Michael Shell}
% or if you just want to reserve a space for a photo:

%\begin{IEEEbiography}[{\includegraphics[width=1in,height=1.25in,clip,keepaspectratio]{picture}}]{John Doe}
%\blindtext
%\end{IEEEbiography}

% You can push biographies down or up by placing
% a \vfill before or after them. The appropriate
% use of \vfill depends on what kind of text is
% on the last page and whether or not the columns
% are being equalized.

%\vfill

% Can be used to pull up biographies so that the bottom of the last one
% is flush with the other column.
%\enlargethispage{-5in}

%De Dnu72 - Trabajo propio, CC BY-SA 3.0, https://commons.wikimedia.org/w/index.php?curid=23851283

% that's all folks
\end{document}


